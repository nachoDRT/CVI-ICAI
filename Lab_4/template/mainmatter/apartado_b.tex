\chapter{Apartado B: \textbf{Flujo Óptico}}
\label{chapter:tarea_b}

\section*{Tarea B.1: Configuración del Flujo Óptico}
\phantomsection
\addcontentsline{toc}{section}{Tarea B.1: Configuración del Flujo Óptico}
Implemente el flujo óptico de Lucas-Kanade, útil para detectar el movimiento pixel a pixel.

\section*{Tarea B.2: Detección de Puntos de Interés}
\phantomsection
\addcontentsline{toc}{section}{Tarea B.2: Detección de Puntos de Interés}
Configure los puntos de interés iniciales en el primer frame.

\section*{Tarea B.3: Cálculo y Visualización del Flujo Óptico}
\phantomsection
\addcontentsline{toc}{section}{Tarea B.3: Cálculo y Visualización del Flujo Óptico}
Calcule el flujo óptico y visualice el movimiento de cada punto de interés en los siguientes frames.

\section*{Preguntas}
\addcontentsline{toc}{section}{Preguntas}

\vspace{5mm}
\begin{tcolorbox}[colback=gray!10, colframe=gray!30, coltitle=black, title=Pregunta B.1, halign=left]
¿Qué efecto tiene el parámetro \texttt{winSize} en la precisión del flujo óptico?
\end{tcolorbox}


\vspace{5mm}
\begin{tcolorbox}[colback=gray!10, colframe=gray!30, coltitle=black, title=Pregunta B.2, halign=left]
¿Cómo influye el parámetro \texttt{qualityLevel} en la función \texttt{cv2.goodFeaturesToTrack} al detectar puntos de interés?
\end{tcolorbox}